\documentclass{article}

% if you need to pass options to natbib, use, e.g.:
% \PassOptionsToPackage{numbers, compress}{natbib}
% before loading rl_project.

% to compile a camera-ready version, add the [final] option, e.g.:
 \usepackage[final]{rl_project}

% to avoid loading the natbib package, add option nonatbib:
% \usepackage[nonatbib]{rl_project}

\usepackage[utf8]{inputenc} % allow utf-8 input
\usepackage[T1]{fontenc}    % use 8-bit T1 fonts
\usepackage{hyperref}       % hyperlinks
\usepackage{url}            % simple URL typesetting
\usepackage{booktabs}       % professional-quality tables
\usepackage{amsfonts}       % blackboard math symbols
\usepackage{nicefrac}       % compact symbols for 1/2, etc.
\usepackage{microtype}      % microtypography
\usepackage{graphicx}


% Give your project report an appropriate title!

\title{RL Project Template}
% Your report should be written using the provided LaTeX template, and should be no longer than seven pages including figures but excluding references and appendices. The content of all figures, including any embedded text, should be clearly legible; you will lose marks for figures and text that are too small to view comfortably. You should not modify the provided LaTeX template. Any appendices should be clearly referenced in the main body of your report. All sources should be referenced appropriately. Each group should submit a single report. This coursework is not marked anonymously.

% The \author macro works with any number of authors. There are two
% commands used to separate the names and addresses of multiple
% authors: \And and \AND.
%
% Using \And between authors leaves it to LaTeX to determine where to
% break the lines. Using \AND forces a line break at that point. So,
% if LaTeX puts 3 of 4 authors names on the first line, and the last
% on the second line, try using \AND instead of \And before the third
% author name.

\author{
  David Hood, Fiora MacPherson, Loving-Grace Mawire, Vanisha Oree
  \\
  Department of Computer Science\\
  University of Bath\\
  Bath, BA2 7AY \\
  \texttt{dh2155@bath.ac.uk} \\
  \texttt{fm683@bath.ac.uk} \\
  \texttt{lgrm21@bath.ac.uk} \\
  \texttt{vo256@bath.ac.uk} \\
  %% examples of more authors
  %% \And
  %% Coauthor \\
  %% Affiliation \\
  %% Address \\
  %% \texttt{email} \\
  %% \AND
  %% Coauthor \\
  %% Affiliation \\
  %% Address \\
  %% \texttt{email} \\
  %% \And
  %% Coauthor \\
  %% Affiliation \\
  %% Address \\
  %% \texttt{email} \\
  %% \And
  %% Coauthor \\
  %% Affiliation \\
  %% Address \\
  %% \texttt{email} \\
}

\begin{document}

\maketitle

\section{Problem Definition}
% A clear, precise and concise description of your chosen problem, including the states, actions, transition dynamics, and the reward function. You will lose marks for an unclear, incorrect, or incomplete problem definition. You should also discuss the difficulty of your chosen problem and justify why it cannot be solved effectively using tabular reinforcement learning methods.

\section{Background}
% A discussion of reinforcement learning methods that may be effective at solving your chosen problem, their strengths and weaknesses for your chosen problem, and any existing results in the scientific literature (or publicly available online) on your chosen problem or similar problems.

\section{Method}
% A description of the method(s) used to solve your chosen problem, an explanation of how these methods work (in your own words), and an explanation of why you chose these specific methods.

\section{Results}
% A presentation of your results, showing how quickly and how well your agent(s) learn (i.e., improve their policies). Include informative baselines for comparison (e.g. the best possible performance, the performance of an average human, or the performance of an agent that selects actions randomly).

\section{Discussion}
% An evaluation of how well you solved your chosen problem.

\section{Future Work}
% A discussion of potential future work you would complete if you had more time.

\section{Personal Experience}
% A discussion of your personal experience with the project, such as difficulties or pleasant surprises you encountered while completing it.


\section*{References}
% 
Mnih, V., Kavukcuoglu, K., Silver, D., Rusu, A.A., Veness, J., Bellemare, M.G., Graves, A., Riedmiller, M., Fidjeland, A.K., Ostrovski, G. and Petersen, S., 2015. Human-level control through deep reinforcement learning. Nature, 518(7540), pp.529-533

\small

\normalsize
\newpage
\section*{Appendices}
% Appendices may include (1) a detailed description of the problem domain, including the states, actions, reward function, and transition dynamics; (2) all experimental details so that the reader can fully replicate your experiments; (3) how you selected your hyperparameters (if applicable); and (4) any additional supporting results that you could not include in the main body of your report. Note that your appendices should be clearly referenced in the main body of your report.
If you have additional content that you would like to include in the appendices, please do so here.
There is no limit to the length of your appendices, but we are not obliged to read them in their entirety while marking. The main body of your report should contain all essential information, and content in the appendices should be clearly referenced where it's needed elsewhere.
\subsection*{Appendix A: Example Appendix 1}
\subsection*{Appendix B: Example Appendix 2}

\end{document}
